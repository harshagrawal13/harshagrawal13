\RequirePackage[utf8]{inputenc}
\documentclass[10pt,a4paper,ragged2e]{altacv}

\geometry{left=1.5cm,right=10cm,marginparwidth=7.7cm,marginparsep=0.8cm,top=1.25cm,bottom=1.25cm}
\usepackage[T1]{fontenc}
\usepackage[default]{lato}

\usepackage{hyperref}
\hypersetup{
    colorlinks=true,
    urlcolor=black,
}

\usepackage[official]{eurosym}


\definecolor{VividPurple}{HTML}{000000}
\definecolor{SlateGrey}{HTML}{2E2E2E}
\definecolor{LightGrey}{HTML}{2E2E2E}
\colorlet{heading}{VividPurple}
\colorlet{accent}{VividPurple}
\colorlet{emphasis}{SlateGrey}
\colorlet{body}{LightGrey}

%bullets & rating marker
\renewcommand{\itemmarker}{{\small\textbullet}}
\renewcommand{\ratingmarker}{\faCircle}

\begin{document}
\name{Harsh Agrawal}
\tagline{<designing proteins>}
% Add Profile Picture here
% \photo{3.3cm}{profile.jpg}

\personalinfo{
    % \homepage{atom51.com}
    \email{harshagrawal.1312@gmail.com}
    \github{github.com/harshagrawal13}
    \twitter{https://x.com/harshag1312}
}

\skillsinfo{
    \code{Machine Learning, Computational Biology, Medicine, and cooking great pasta.}
}
\begin{fullwidth}
    \makecvheader
\end{fullwidth}

%% Depending on your tastes, you may want to make fonts of itemize environments slightly smaller
\AtBeginEnvironment{itemize}{\footnotesize}


\marginpar{
    \bigskip%
    \vspace*{
        \dimexpr1pt-\baselineskip
    }
    \raggedright{
        \cvsection{Research}
        \cvevent{\textbf{JESPR} - Energy Based Modelling for Proteins}{}{  May 2023 - December 2023}{}
        \begin{itemize}
            \item Devised and implemented a contrastive learning approach for the pre-training of foundational protein models, aimed at learning high-level semantic representations between sequence and structural spaces to facilitate bi-directional mapping.
            \item Also implemented sharding, activation checkpointing, and other memory-saving techniques for scalable multi-GPU training of models with 500M+ parameters.
            \item Currently evaluating zero-shot performance on protein function benchmarks such as solubility, stability, etc. (from https://arxiv.org/abs/2206.02096).
            \item Code: \href{https://github.com/atom-51/jespr}{github.com/atom-51/jespr}
        \end{itemize}

        \divider

        \cvevent{\textbf{Cancer-eX} - Detecting Pancreatic Cancer early}{}{Aug 2018 - August 2021}{}
        \begin{itemize}
            \item Studied a salivary biomarker to detect Pancreatic and other Gastrointestinal Cancers early and proposed a lateral flow assay-based system for early detection.
            \item Designed and calibrated a competitive ELISA protocol from scratch. \textit{(The outcome was partially successful)}
            \item Also Designed a simple Risk-assessment tool for Pancreatic Cancer using etiological risk factors by factoring their respective relative-risk (RR) values obtained via independent studies.
        \end{itemize}

        \cvsection{Achievements}
        \begin{itemize}
            \cvitem{{Rise Global Winner 2021}}{July 2022}{Selected as one of the 100 global winners by the Rhodes Trust and Schmidt Futures. A post-secondary academic scholarship and an invitation to a residential summit were provided.},

            \cvitem{{Bal Shakti Puraskar Recipient 2020}}{Jan 2020}{Highest honor bestowed to an Indian below 18 by the \textbf{President of India} at the President's House.}

            \cvitem{Global Teen Leader 2020}{Jan 2020}{Selected as one of the 35 teens all around the world by Three Dot Dash to participate at the Virtual Just Peace Summit 2020.}

            \cvitem{Gold Medal, Genius Olympiad}{June 2020}{Won gold medal in the category of sciences at one of the largest high-school interdisciplinary competitions. It witnessed participation from ~1600 exhibits from ~80 countries.}

            \cvitem{Regeneron ISEF Finalist}{May 2020}{Qualified as one of the 20 team India finalists for ISEF 2020. The competition was canceled due to the pandemic.}

            \cvitem{Regional Finalist, Google Science Fair}{May 2019}{Selected as one of the top 100 exhibits all around the world at the Google Science Fair.}

            \cvitem{Third Award, Spark Teen Accelerator Program}{June 2020}{Won third award globally (+ USD 1500) for CANCER-EX along with acceptance to their accelerator program.}

            \cvitem{Grand Award at IRIS Nationals}{Jan 2020}{Granted in the category of biochemistry at the biggest science fair in the country.}

            \cvitem{CSIR Innovation Award}{Oct 2020}{Highest research honor bestowed to an Indian under 18 by the Science \& Technology Minister of India.}

        \end{itemize}

    }
}

\cvsection{Education}

\cvevent{Masters in Engineering}{Imperial College London}{October 2022 - Present}{}
\begin{itemize}
    \item Pursuing a combined Bachelors + Masters (4 years) degree in Molecular Bioengineering. Tuition + living + travel \textbf{completed sponsored} by RISE scholarship.
    \item Recipient of the Department Scholarship of \textbf{1000 GBP} per year.
\end{itemize}
\divider
\cvevent{High School Dimploma}{Bharatiya Vidya Bhavans, Raipur}{April 2018 - May 2021}{}
\begin{itemize}
    \item 12th Grade: 96\% (Class Rank: 2 / 20)
    \item English (99/100), Math (98/100), Biology (91/100), Physics (97/100), Chemistry (92/100), and Computer Science (99/100).
\end{itemize}

\cvsection{Experience}
\cvevent{Research Scientist}{Lab. of Medical Sciences, Imperial College London}{June 204 - Present}{}
\begin{itemize}
    \item {Working with \href{https://lms.mrc.ac.uk/research-group/synthetic-biology/}{\textbf{Dr. Karen Sarkisyan}} on re-purposing large protein toxins to induce novel binding affinity for different targets. The aim is to design high affinity binders that bind with nano-molar affinity to specific transmembrane receptors in the guts of small insects like c. \textit{elegans}}.
    \item {I fully designed the entire computational pipeline for generation and screening, from evaluating binding interfaces using specific PPI databases to generating new proteins (including backbones and sequences) with state-of-the-art diffusion models. This process included Rosetta-based relaxation and large-scale AlphaFold2-based screening, using metrics such as DockQ, iPTM, and iPAE.}
    \item {I also developed the necessary MLOps infrastructure from the ground up, scaling throughput from $\approx$10 to >500 designs per day (each protein > 1,000 residues) by distributing inference across 4 HPC servers and approximately 40 GPUs.}
\end{itemize}
\divider
\cvevent{Lead Data Scientist}{Paraswap.io}{May 2021 - June 2022}{}
\begin{itemize}
    \item {Paraswap is the second largest multi-chain Decentralized Exchange aggregator for crypto-currencies. I worked as their lead data scientist primarily responsible for developing a public dashboard to aggregate Paraswap's statistics such as Users, Volume, and Fees along with other tasks.}
    \item {The Dashboard indexes ~\$100M in daily volume across 6 blockchains and >\$55B in total volume. Public Dashboard: \href{https://www.paraswap.io/dashboard}{paraswap.io/dashboard}}
    \item {Also developed multiple private dashboards to simulate blockchain transactions, used to make crucial decisions.}
\end{itemize}
\divider
\cvevent{Hackathon - The Gawler Challenge}{MineNow}{Apr 2020 - Jul 2020}{}
\begin{itemize}
    \item The Gawler Challenge was a global online competition from the Government of South Australia to find mineralization using Machine Learning.
    \item Created three separate machine learning models to predict novel mineralized sites in the Gawler region of South Australia using the existing geo-physical and geo-chemical data-sets. A total of ~300,000 new mineralized points were predicted along with their size and predicted ore/mineral.
    \item Code: \href{https://github.com/harshagrawal13/Gawler-Unearthed}{github.com/harshagrawal13/Gawler-Unearthed}
\end{itemize}

\clearpage
% Anything that I add after this page goes to the next page
\end{document}
