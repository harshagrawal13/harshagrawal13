\RequirePackage[utf8]{inputenc}
\documentclass[10pt,a4paper,ragged2e]{altacv}

\geometry{left=1.5cm,right=10cm,marginparwidth=7.7cm,marginparsep=0.8cm,top=1.25cm,bottom=1.25cm}
\usepackage[T1]{fontenc}
\usepackage[default]{lato}

\usepackage{hyperref}
\hypersetup{
    colorlinks=true,
    urlcolor=black,
}

\usepackage[official]{eurosym}


\definecolor{VividPurple}{HTML}{000000}
\definecolor{SlateGrey}{HTML}{2E2E2E}
\definecolor{LightGrey}{HTML}{2E2E2E}
\colorlet{heading}{VividPurple}
\colorlet{accent}{VividPurple}
\colorlet{emphasis}{SlateGrey}
\colorlet{body}{LightGrey}

%bullets & rating marker
\renewcommand{\itemmarker}{{\small\textbullet}}
\renewcommand{\ratingmarker}{\faCircle}

\begin{document}
\name{Harsh Agrawal}
\tagline{<designing proteins>}

\personalinfo{
    % \homepage{atom51.com}
    \email{harshagrawal.1312@gmail.com}
    \github{github.com/harshagrawal13}
    \twitter{https://x.com/harshag1312}
}

% Maybe I should add my current experience here.
\skillsinfo{
    \code{ML Engineering, Protein Design, Computational Biology.}
}
\begin{fullwidth}
    \makecvheader
\end{fullwidth}

%% Depending on your tastes, you may want to make fonts of itemize environments slightly smaller
\AtBeginEnvironment{itemize}{\footnotesize}


\marginpar{
    \bigskip%
    \vspace*{
        \dimexpr1pt-\baselineskip
    }
    \raggedright{
        \cvsection{Personal Projects}
        \cvevent{\textbf{JESPR} - Energy Based Modelling for Proteins}{}{  May 2023 - December 2023}{}
        \begin{itemize}
            \item Devised and implemented a contrastive learning approach for the pre-training of foundational protein models, aimed at learning high-level semantic representations between sequence and structural spaces to facilitate bi-directional mapping.
            \item Also implemented sharding, activation checkpointing, and other memory-saving techniques for scalable multi-GPU training of models with 500M+ parameters.
            \item Currently evaluating zero-shot performance on protein function benchmarks such as solubility, stability, etc. (from https://arxiv.org/abs/2206.02096).
            \item Code: \href{https://github.com/atom-51/jespr}{github.com/atom-51/jespr}
        \end{itemize}

        \divider

        \cvevent{\textbf{Cancer-eX} - Detecting Pancreatic Cancer early}{}{Aug 2018 - August 2021}{}
        \begin{itemize}
            \item Studied a salivary biomarker to detect Pancreatic and other Gastrointestinal Cancers early and proposed a lateral flow assay-based system for early detection.
            \item Designed and calibrated a competitive ELISA protocol from scratch. \textit{(The outcome was partially successful)}
            \item Also Designed a simple Risk-assessment tool for Pancreatic Cancer using etiological risk factors by factoring their respective relative-risk (RR) values obtained via independent studies.
        \end{itemize}

        \divider

        \cvevent{\textbf{MineNow} - The Gawler Challenge}{}{\\ Apr 2020 - Jul 2020}{}
        \begin{itemize}
            \item The Gawler Challenge was a global online competition from the Government of South Australia to find mineralization using Machine Learning.
            \item Created three separate machine learning models to predict novel mineralized sites in the Gawler region of South Australia using the existing geo-physical and geo-chemical data-sets. A total of ~300,000 new mineralized points were predicted along with their size and predicted ore/mineral.
            \item Code: \href{https://github.com/harshagrawal13/Gawler-Unearthed}{github.com/harshagrawal13/Gawler-Unearthed}
        \end{itemize}

        \cvsection{Achievements}
        \begin{itemize}
            \cvitem{\href{https://www.rhodeshouse.ox.ac.uk/about/rise/}{\textbf{Rise Global Winner 2021}}}{July 2022}{Selected as one of the 100 global winners for one of the largest high-school scholarships in the world by the Rhodes Trust and Schmidt Futures.}

            \cvitem{{Bal Shakti Puraskar Recipient 2020}}{Jan 2020}{Highest honor bestowed to an Indian below 18 by the \textbf{President of India} at the President's House. Awarded for my work on Cancer-eX.}

            \cvitem{Global Teen Leader 2020}{Jan 2020}{Selected as one of the 35 teens all around the world by Three Dot Dash to participate at the Virtual Just Peace Summit 2020. Was part of a youth delegation to speak at the World Economic Forum 2024 in Davos, Switzerland.}

            \cvitem{Gold Medal, Genius Olympiad}{June 2020}{Won in the category of sciences at one of the largest high-school interdisciplinary competitions. It witnessed participation from ~1600 exhibits from ~80 countries.}

            \cvitem{Regional Finalist, Google Science Fair}{May 2019}{Selected as one of the top 100 exhibits all around the world at the Google Science Fair.}

        \end{itemize}

    }
}

\cvsection{Education}

\cvevent{Masters in Engineering}{Imperial College London}{October 2022 - Present}{}
\begin{itemize}
    \item Pursuing a Bachelor's + Master's (4 years) in Molecular Bioengineering. Fully sponsored by the \textbf{RISE scholarship}.
    \item Recipient of the Department Scholarship of \textbf{1000 GBP} per year.
    \item Key modules include Advanced Synthetic Biology, Tissue Engineering, Biomaterials, Reinforcement Learning, Thermodynamics, etc.
\end{itemize}
\divider
\cvevent{High School Dimploma}{Bharatiya Vidya Bhavans, Raipur}{April 2018 - May 2021}{}
\begin{itemize}
    \item 12th Grade: 96\% | Class Rank: 2 / 20
\end{itemize}

\cvsection{Experience}
\cvevent{Research Scientist}{Softnanolab, Imperial College London}{Feb 2025 - Present}{}
\begin{itemize}
    \item {Currently developing a Protein Language Model tailored to multimeric proteins, trained on a large-scale synthetic multimer dataset. This work is supported by a \textbf{10K B200 GPU-hour Isambard Grant} (\href{https://blogs.nvidia.com/blog/isambard-ai/}{Isambard AI}) and a \textbf{16K A100 GPU-hour NVIDIA Grant}, and is conducted under the supervision of \href{https://profiles.imperial.ac.uk/s.angioletti-uberti/about}{\textbf{Dr. Stefano Angioletti Uberti}}.}

    \item {Previously led the development of \textit{ecstasy}, an open-source, comprehensive benchmarking framework for multimeric structure prediction methods (AF3, Boltz2, ESMFold, \emph{et al.}), designed to assess robustness to chain-order permutations and consistency of predicted multimer assemblies.}
\end{itemize}
\divider
\cvevent{Research Scientist}{Lab. of Medical Sciences, MRC Imperial College London}{May 2024 - Jan 2025}{}
\begin{itemize}
    \item {Engineered novel, nanomolar-affinity binders for \textit{C. elegans} transmembrane receptors by re-purposing large protein toxins and designing new target-specific affinities. (\textit{In collaboration with} \href{https://lms.mrc.ac.uk/research-group/synthetic-biology/}{\textbf{Dr. Karen Sarkisyan}} \textit{and} \href{https://tierpsy.com/andre}{\textbf{Andre Brown}}.)}
    \item {Designed the end-to-end computational pipeline for protein design and screening: evaluated binding interfaces using PPI databases, generated novel proteins (backbones and sequences) with state-of-the-art diffusion models (e.g., RFDiffusion), performed Rosetta relaxation, and conducted large-scale AlphaFold2 screening using metrics such as DockQ, iPTM, and iPAE.}
    \item {Built MLOps infrastructure from the ground up, scaling in-silico throughput by \textbf{50x} (from $\sim$10 to $>$500 designs/day) via distributed asynchronous inference across four heterogeneous compute clusters ($>$40 GPUs).}
    \item {Selected 10 designs for experimental validation by optimizing in-silico metrics (e.g., DockQ, iPTM) across a screening pool of $\approx$10K candidates.}
\end{itemize}
\divider
\cvevent{Lead Data Scientist}{Velora (formerly Paraswap.io)}{May 2021 - June 2022}{}
\begin{itemize}
    \item {Built the public analytics pipeline and \href{https://dashboard.velora.xyz/public/dashboard/bb8b5007-ecc3-44e9-bdbe-b4b7a5273eb4}{\textbf{dashboard}} for \textbf{Velora}, the second-largest DEX aggregator, architecting a robust system indexing 33M+ transactions and \$140B+ in trading volume across 5 blockchains.}
    \item {Co-led the upgrade of the full data infrastructure from Heroku to AWS, deploying a scalable Redshift + MongoDB warehouse to support low-latency querying, reliable ingestion, and long-term storage of high-volume on-chain data.}
    \item {Developed internal dashboards powered by \href{https://tenderly.co/}{\textbf{Tenderly}} simulations to model transaction outcomes and MEV behavior, directly informing decisions on routing optimisation, onboarding new DEXs, and smart contract improvements.}
\end{itemize}
\clearpage
% Anything that I add after this page goes to the next page
\end{document}
